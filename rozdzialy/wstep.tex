\chapter{Wstęp}

Niniejszy szablon zawiera propozycję formatowania pracy dyplomowej, zgodne z wytycznymi z \href{https://lex.polsl.pl/423-lista/d/20505/5/}{Zarządzenia nr 54/2022}. Szablon dostępny jest na licencji \texttt{GNU LGPL v2.1}. Stosowanie, modyfikowanie i rozpowszechnianie niniejszego szablonu jest dozwolone w dowolnym celu, przy czym sam szablon traktowany jest jak zewnętrzna biblioteka. Plik loga \verb|polsl_logo.png|, który znajduje się w folderze \texttt{obrazki}, jest własnością Politechniki Śląskiej i jego stosowanie regulowane jest przepisami zawartymi w \href{https://www.polsl.pl/siwps/logo-2/}{Księdze Znaku}. Uwagi proszę zgłaszać na adres autora: \href{mailto:lukasz.drozdz@polsl.pl}{lukasz.drozdz@polsl.pl}.

Szablon złożony jest z następujących katalogów:
\begin{itemize}
\item \texttt{rozdzialy} -- tutaj zaleca się umieszczać kolejne pliki rozdziałów,
\item \texttt{obrazki}   -- tutaj zaleca się umieszczanie obrazków i rysunków,
\item \texttt{dodatki}   -- tutaj znajdują się streszczenia i bibliografia,
\item \texttt{budowa}    -- tutaj pojawi się wygenerowany plik \texttt{PDF}.
\end{itemize}

W katalogu głównym natomiast znaleźć można pliki:
\begin{itemize}
\item \texttt{build.sh}       -- skrypt stosowany do wygenerowania pliku \texttt{PDF},
\item \texttt{thesis.cls}     -- klasa szablonu dokumentu (nie należy edytować),
\item \texttt{thesis.tex}     -- główny plik dokumentu (należy uzupełnić),
\item \texttt{thesis.xmpdata} -- metryka dokumentu (należy uzupełnić).
\end{itemize}

W pliku \texttt{thesis.tex} należy uzupełnić dane dotyczące pracy, zgodnie z komentarzami w pliku, a następnie dodać kolejne rozdziały pracy. Dodatkowo dane \texttt{XMP} dla dokumentu \texttt{PDF} należy uzupełnić w pliku \texttt{thesis.xmpdata}. Pliku \texttt{thesis.cls} nie zaleca się edytować -- można to jednak robić, jeśli konieczna jest edycja stylu dokumentu. Dokument \texttt{PDF} generowany jest w formacie \texttt{PDF/A-3u}. Marginesy dokumentu ustawione są tak, aby wewnętrzny margines był większy od zewnętrznego, co umożliwia zbindowanie pracy przy jednoczesnym zachowaniu jej czytelności. Marginesy edytować można w pliku \texttt{thesis.cls}.
