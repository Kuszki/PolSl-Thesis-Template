\chapter{Wstęp}

Niniejszy szablon zawiera formaty nagłówków, stronę tytułową jak i propozycje spisu treści, podpisów pod rysunkami, wzory tabel, itp. przeznaczone jako wzór przy formatowaniu pracy dyplomowej.

Szablon złożony jest z następujących katalogów:
\begin{itemize}
\item \texttt{rozdzialy} -- tutaj zaleca się umieszczać kolejne pliki rozdziałów,
\item \texttt{obrazki}   -- tutaj zaleca się umieszczanie obrazków i rysunków,
\item \texttt{dodatki}   -- tutaj znajdują się streszczenia i bibliografia,
\item \texttt{budowa}    -- tutaj pojawi się wygenerowany plik \texttt{PDF}.
\end{itemize}

W katalogu głównym natomiast znaleźć można pliki:
\begin{itemize}
\item \texttt{build.sh}       -- skrypt stosowany do wygenerowania pliku \texttt{PDF},
\item \texttt{thesis.cls}     -- klasa szablonu dokumentu (nie należy edytować),
\item \texttt{thesis.tex}     -- główny plik dokumentu (należy uzupełnić),
\item \texttt{thesis.xmpdata} -- metryka dokumentu (należy uzupełnić).
\end{itemize}

W pliku \texttt{thesis.tex} należy uzupełnić dane dotyczące pracy, zgodnie z komentarzami w pliku, a następnie dodać kolejne rozdziały pracy. Dodatkowo dane \texttt{XMP} dla dokumentu \texttt{PDF} należy uzupełnić w pliku \texttt{thesis.xmpdata}. Pliku \texttt{thesis.cls} nie zaleca się edytować -- można to jednak robić, jeśli konieczna jest edycja stylu dokumentu.

Dokument \texttt{PDF} generowany jest w formacie \texttt{PDF/A-3u}. Marginesy dokumentu ustawione są tak, aby wewnętrzny margines był większy od zewnętrznego, co umożliwia zbindowanie pracy przy jednoczesnym zachowaniu jej czytelności. W celu uzyskania wersji z równymi marginesami (np. w celu zapisu jej na płytę) należy zamienić \verb|inner = 3.0cm, outer = 2.0cm| na \verb|inner = 2.5cm, outer = 2.5cm| w pliku \texttt{thesis.cls}. Ogólny układ pracy nie ulegnie zmianie.
