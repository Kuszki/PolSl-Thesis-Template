\chapter{Kompilacja dokumentu}

Niniejszy szablon może być stosowany bezpośrednio w serwisie \texttt{Overleaf}. Należy jednak koniecznie zmienić stosowany kompilator z \texttt{pdfTeX} na \texttt{LuaTeX}. Należy zauważyć, że \texttt{pdfTeX} jest przestarzały i nie zapewnia wielu nowoczesnych udogodnień, w tym opcji \mintinline{latex}{\babelprovide[transforms = oneletter.nobreak]{polish}} oraz generowania pliku w standardzie \texttt{PDF/A-3u}.

Zaleca się jednak, z uwagi na ochronę własności intelektualnej i zapobieganie analizowaniu dokumentu przez obce podmioty, instalację środowiska \LaTeX{} na własnym komputerze. Najlepszym wyjściem jest instalacja pakietu \texttt{texlive}, który dostępny jest w zasadzie dla każdej dystrybucji systemu operacyjnego z rodziny \texttt{GNU/Linux}. W przypadku platformy \texttt{MacOS} dostępny jest pakiet \texttt{MacTeX}, natomiast dla platformy \texttt{Windows} istnieje możliwość instalacji portu programu \texttt{TeX Live}.

Dysponując powłoką zgodną z \texttt{bash} istnieje możliwość uruchomienia skryptu \texttt{build.sh}, który kolejno dokona konwersji obrazów z formatu programu \texttt{LibreOffice} oraz formatu \texttt{SVG} do formatu \texttt{PDF}, a następnie wygeneruje w folderze \texttt{budowa} plik \texttt{thesis.pdf} z gotowym dokumentem. Omawiany skrypt przedstawiono w listingu~\ref{lst:build}, przy czym do jego uruchomienia wymagane są programy \texttt{LibreOffice} oraz \texttt{Inkscape}. Wskazane programy są konieczne jedynie w celu konwersji obrazów do formatu \texttt{PDF}.

\begin{listing}[hbt!]
\inputminted[linenos, breaklines]{bash}{build.sh}
\makecaption{lst:build}{Skrypt \texttt{bash} ułatwiający kompilację dokumentu}
\end{listing}
